
\section{Hachage}

    On a choisit d'aussi implementer le haschage MD5 en OCaml alors qu'un langage de programation plus bas niveau comme le C qui permet de caster les valeurs pour interpreter directement une meme valeurs physique de manière differente. 

    \subsection {representation des valeurs}

    On choisit d'avoir comme entré un argument de type string qu'on convertie ensuite en une liste de tableau de 16 entier 32. 
    on fais ce choix car le MD5 travaille sur des bloques de 512 bits soit 16 entier 32 bits. A chaque tour de l'algorithme on travaille donc sur un seul element de cette liste de tableaux.

    \subsection{4.2 traitement du messages} 
    \subsubsection{transformation de la chaine }
    
    pour transformer notre string en int32 on utilise une fonction du module String qui permet de decoder bit a bit des int32 depuis des string 
    
\begin{lstlisting}
    val get_int32_le : string -> int -> int32
\end{lstlisting}  

    cette fonction prend la chaine et l'indice de debut de l'entier a decoder en argument et renvoit une erreur si il reste moins de 32 bit a decoder (moins de 4 charactères).

    de plus la chaine doit etre paddé pour etre divisible en bloque de 64 octets. 
    on doit ajouter un bit a 1 pour marqué la fin de la chaine puis paddé avec des 0 jusqu'a ce qu'il reste 2 octets dans lesquels on stock la taille de la chaine en bit sut 64 bits. 

    
    on ne va pas detailler cette transformation trivial a partir du moment ou on peu de transformer notre chaine en entier.


    \subsubsection{implementation du md5}
    
    on implemente le md5 en suivant l'algorithme classique. on notera juste l'ajout de la fonction finish
\begin{lstlisting}
let finish (entre : int32) : int32 =
  let res = 0l in
  let res = Int32.logor res (Int32.shift_left (Int32.logand entre 0x000000FFl) 24) in 
  let res = Int32.logor res (Int32.shift_left (Int32.logand entre 0x0000FF00l) 8) in 
  let res = Int32.logor res (Int32.shift_right_logical (Int32.logand entre 0x00FF0000l) 8) in 
  let res = Int32.logor res (Int32.shift_right_logical (Int32.logand entre 0xFF000000l) 24) in 
  res
;;
\end{lstlisting} 

    on execute cette fonction sur les 4 int32 qui compose le int128 renvoyé par la boucle du md5 elle permet d'inverser les octets de l'entier, ici pour le ramener en little endian.